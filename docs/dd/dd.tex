\documentclass[a4paper, 11pt, parskip=half]{scrreprt}

\usepackage[automark]{scrlayer-scrpage}         % Headings
\usepackage{DejaVuSerif}                        % Default font
\usepackage{DejaVuSans}                         % Headers font
\usepackage{sourcecodepro}                      % Monospace font
\usepackage[utf8]{inputenc}                     % Input encoding
\usepackage[T1]{fontenc}                        % Output encoding
\usepackage{graphicx}                           % Pictures
\usepackage{listings}                           % Code snippets
\usepackage{hyperref}                           % Make strings clickable
\usepackage{amsmath,amsfonts,stmaryrd,amssymb}  % Math packages
\usepackage{amsthm}                             % Definitions, theorems, corollaries, ...
\usepackage{mathtools}                          % Math stuff
\usepackage[usenames,dvipsnames,table]{xcolor}  % Colors
\usepackage[toc,page]{appendix}                 % Support for appendices
\usepackage[chapter]{algorithm}                 % Pseudocode headers
\usepackage{algpseudocode}                      % Pseudocode
\usepackage{pdfpages}                           % Embed external pdf
\usepackage{wrapfig}                            % Wrap text around figures
\usepackage{multicol}                           % Used for multicolumn itemize
\usepackage{multirow}                           % Used for multirow in tables
\usepackage{longtable}                          % Tables can spread over multiple pages
\usepackage{enumerate}                          % Custom item numbers for enumerations
\usepackage[framemethod=tikz]{mdframed}         % Allows defining custom boxed/framed environments
\usepackage{tocloft}                            % TOC spacing


% TODO: REMOVE
\usepackage{lipsum}
\usepackage{blindtext}



%----------------------------------------------------------------------------------------
%	DOCUMENT SETTINGS
%----------------------------------------------------------------------------------------

\areaset{17cm}{22.5cm}              % Set page width and height
\graphicspath{{./figures/}}         % Set path for figures
\setlength{\cftchapnumwidth}{2em}   % Set chapter numwidth
\setlength{\cftsecnumwidth}{3em}    % Set section numwidth
\setlength{\cftsubsecnumwidth}{4em} % Set subsection numwidth
\hypersetup{linktoc=all}            % Set TOC clickable

\lstset{ 
    belowcaptionskip=\baselineskip,
    aboveskip=\baselineskip,
    breaklines=true,
    frame=l,
    xleftmargin=0.5in,
    showstringspaces=false,
    basicstyle=\footnotesize\ttfamily,
    keywordstyle=\bfseries\color{green!40!black},
    commentstyle=\color{MidnightBlue},
    stringstyle=\color{BrickRed},
    numberstyle=\color{Cyan!50!Black},
    numbers=left,
    tabsize=4
}

\theoremstyle{definition}
\newtheorem{definition}{Definition}[chapter]
\newtheorem{theorem}{Theorem}[chapter]
\newtheorem{corollary}{Corollary}[theorem]


%----------------------------------------------------------------------------------------
%	COMMAND LINE ENVIRONMENT
%----------------------------------------------------------------------------------------

% Usage:
% \begin{commandline}
%	\begin{verbatim}
%		$ ls
%		
%		Applications	Desktop	...
%	\end{verbatim}
% \end{commandline}

\mdfdefinestyle{commandline}{
	leftmargin=10pt,
	rightmargin=10pt,
	innerleftmargin=15pt,
	middlelinecolor=black!50!white,
	middlelinewidth=2pt,
	frametitlerule=false,
	backgroundcolor=black!5!white,
	frametitle={Command line},
	frametitlefont={\normalfont\ttfamily\color{white}\hspace{-1em}},
	frametitlebackgroundcolor=black!50!white,
	nobreak,
}

% Define a custom environment for command-line snapshots
\newenvironment{commandline}{
	\medskip
	\begin{mdframed}[style=commandline]
	\footnotesize
}{
	\end{mdframed}
	\medskip
}


%----------------------------------------------------------------------------------------
%	FILE CONTENTS ENVIRONMENT
%----------------------------------------------------------------------------------------

% Usage:
% \begin{file}[optional filename, defaults to "File"]
%	File contents, for example, with a listings environment
% \end{file}

\mdfdefinestyle{file}{
	innertopmargin=1.6\baselineskip,
	innerbottommargin=0.28\baselineskip,
	topline=false, bottomline=false,
	leftline=false, rightline=false,
	leftmargin=2cm,
	rightmargin=2cm,
	singleextra={%
		\draw[fill=black!10!white](P)++(0,-1.3em)rectangle(P-|O);
		\node[anchor=north west]
		at(P-|O){\footnotesize\ttfamily\mdfilename};
		%
		\def\l{1.5em}
		\draw(O-|P)++(-\l,0)--++(\l,\l)--(P)--(P-|O)--(O)--cycle;
		\draw(O-|P)++(-\l,0)--++(0,\l)--++(\l,0);
	},
	nobreak,
}

% Define a custom environment for file contents
\newenvironment{file}[1][File]{ % Set the default filename to "File"
	\medskip
	\newcommand{\mdfilename}{#1}
	\begin{mdframed}[style=file]
}{
	\end{mdframed}
	\medskip
}


%----------------------------------------------------------------------------------------
%	NUMBERED QUESTIONS ENVIRONMENT
%----------------------------------------------------------------------------------------

% Usage:
% \begin{question}[optional title]
%	Question contents
% \end{question}

\mdfdefinestyle{question}{
	innertopmargin=1.2\baselineskip,
	innerbottommargin=0.8\baselineskip,
	roundcorner=5pt,
	nobreak,
	singleextra={%
		\draw(P-|O)node[xshift=1em,anchor=west,fill=white,draw,rounded corners=5pt]{%
		Question \theQuestion\questionTitle};
	},
}

\newcounter{Question} % Stores the current question number that gets iterated with each new question

% Define a custom environment for numbered questions
\newenvironment{question}[1][\unskip]{
	\bigskip
	\stepcounter{Question}
	\newcommand{\questionTitle}{~#1}
	\begin{mdframed}[style=question]
}{
	\end{mdframed}
	\medskip
}


%----------------------------------------------------------------------------------------
%	BOXED PARAGRAPH ENVIRONMENT
%----------------------------------------------------------------------------------------

% Usage:
% \begin{boxedpar}[optional title]
%	Question contents
% \end{boxedpar}

\mdfdefinestyle{boxedpar}{
	innertopmargin=1.2\baselineskip,
	innerbottommargin=0.8\baselineskip,
	roundcorner=5pt,
	nobreak,
	singleextra={%
		\draw(P-|O)node[xshift=1em,anchor=west,fill=white,draw,rounded corners=5pt]{%
		\textit{\boxTitle}};
	},
}

% Define a custom environment for numbered questions
\newenvironment{boxedpar}[1][in-depth]{
	\bigskip
	\newcommand{\boxTitle}{#1}
	\begin{mdframed}[style=boxedpar]
}{
	\end{mdframed}
	\medskip
}


%----------------------------------------------------------------------------------------
%	ROUNDED BOX ENVIRONMENT
%----------------------------------------------------------------------------------------

% Usage:
% \begin{roundedbox}
%	Contents
% \end{roundedbox}

\mdfdefinestyle{roundedbox}{
	innertopmargin=0.5\baselineskip,
	innerbottommargin=0.5\baselineskip,
	roundcorner=5pt,
	nobreak,
}

% Define a custom environment for numbered questions
\newenvironment{roundedbox}{
	\bigskip
	\begin{mdframed}[style=roundedbox]
}{
	\end{mdframed}
	\medskip
}


%----------------------------------------------------------------------------------------
%	WARNING TEXT ENVIRONMENT
%----------------------------------------------------------------------------------------

% Usage:
% \begin{warn}[optional title, defaults to "Warning:"]
%	Contents
% \end{warn}

\mdfdefinestyle{warning}{
	topline=false, bottomline=false,
	leftline=false, rightline=false,
	nobreak,
	singleextra={%
		\draw(P-|O)++(-0.5em,0)node(tmp1){};
		\draw(P-|O)++(0.5em,0)node(tmp2){};
		\fill[black,rotate around={45:(P-|O)}](tmp1)rectangle(tmp2);
		\node at(P-|O){\color{white}\scriptsize\textbf !};
		\draw[very thick](P-|O)++(0,-1em)--(O);%--(O-|P);
	}
}

% Define a custom environment for warning text
\newenvironment{warn}[1][Warning:]{ % Set the default warning to "Warning:"
	\medskip
	\begin{mdframed}[style=warning]
		\noindent{\textbf{#1}}
}{
	\end{mdframed}
}


%----------------------------------------------------------------------------------------
%	INFORMATION ENVIRONMENT
%----------------------------------------------------------------------------------------

% Usage:
% \begin{info}[optional title, defaults to "Info:"]
% 	contents
% 	\end{info}

\mdfdefinestyle{info}{%
	topline=false, bottomline=false,
	leftline=false, rightline=false,
	nobreak,
	singleextra={%
		\fill[black](P-|O)circle[radius=0.4em];
		\node at(P-|O){\color{white}\scriptsize\textbf i};
		\draw[very thick](P-|O)++(0,-0.8em)--(O);%--(O-|P);
	}
}

% Define a custom environment for information
\newenvironment{info}[1][Info:]{ % Set the default title to "Info:"
	\medskip
	\begin{mdframed}[style=info]
		\noindent{\textbf{#1}}
}{
	\end{mdframed}
}


%----------------------------------------------------------------------------------------
%	LINEDQUOTE ENVIRONMENT
%----------------------------------------------------------------------------------------

% Usage:
% \begin{linedquote}
% 	contents
% 	\end{linedquote}

\mdfdefinestyle{linedquote}{%
	topline=false, bottomline=false,
	leftline=false, rightline=false,
	nobreak,
	singleextra={%
		\draw[very thick](P-|O)++(0,0)--(O);%--(O-|P);
	}
}

% Define a custom environment
\newenvironment{linedquote}{
	\begin{mdframed}[style=linedquote]
}{
	\end{mdframed}
}


%----------------------------------------------------------------------------------------
%	TITLE PAGE
%----------------------------------------------------------------------------------------

\publishers{
    \begin{figure}[t]
        \centering
        \includegraphics[width=0.9\linewidth, keepaspectratio]{logo}
    \end{figure}
}
\title{Smart Tourist}
\subtitle{Design and Implementation of Mobile Applications\\Design Document}
\date{A.Y. 2019/2020}
\author{Fabio Codiglioni, Alessandro Nichelini}



%----------------------------------------------------------------------------------------
%	DOCUMENT
%----------------------------------------------------------------------------------------

\begin{document}

% Title page and TOC
\pagenumbering{gobble}
\maketitle
%\shipout\null           % Blank page
\pagenumbering{roman}
\tableofcontents
\newpage
\pagenumbering{arabic}


% Body

\chapter{Introduction}
This is the \textit{design document} (DD) of the iOS application "\textbf{Smart Tourist}" developed by \textit{Fabio Codiglioni} and \textit{Alessandro Nichelini} in the context of "Design and Implementation of Mobile Application" course at Politecnico di Milano. \\The document explains the most important design choice we made and the motivations behind them, with specific focus on the Redux like architecture adopted.


\chapter{General Overview}
Smart 

\chapter{Architectural design}

\chapter{Data design}

\chapter{User interface}

\chapter{Notifications}

\chapter{External services and libraries}
\section{Services}
SmartTourist relies almost only on \textit{free data and services}. Here, it follows the description of the external services used.

\begin{itemize}
	\item \textit{\textbf{Wikipedia}}: all displayed attractions are taken from Wikidata knowledge base. The given api is quite basic: attractions entries are retrieved with a SPARQL query embedded in a http API call. Details and places' pictures are retrieved with standard HTTP API calls from different endpoints.
	\item \textit{\textbf{Google}}: Places ratings are the only data retrieved from Google services. This was due to the lack of a reliable free alternative for places' rating. For the simplicity of the task, we decided not to rely on the Google SDK for iOS and to manually make request to the Google API. 
\end{itemize}
\section{Libraries}
The app uses lots of third parties' libraries the can be roughly divided into three kind: architectural libraries, back-end libraries and front-end libraries.


\begin{center}

\begin{tabular}{|c|c|l|}
\hline
\textbf{Kind}                  & \textbf{Library}     & \textbf{Description}                                                                                                                                                                                        \\ \hline
\multirow{2}{*}{Architectural} & Katana               & Katana is a modern Swift framework for writing iOS applications' business logic that are testable and easy to reason about. Katana is inspired by Redux.                                                    \\ \cline{2-3} 
                               & Tempura              & Tempura is a holistic approach to iOS development, it borrows concepts from Redux (through Katana) and MVVM.                                                                                                \\ \hline
\multirow{5}{*}{Back-end}      & DeepDiff             & DeepDiff tells the difference between 2 collections and the changes as edit steps.                                                                                                                          \\ \cline{2-3} 
                               & Fuse                 & Fuse is a super lightweight library which provides a simple way to do fuzzy searching.                                                                                                                      \\ \cline{2-3} 
                               & Alamofire            & Alamofire is an HTTP networking library written in Swift.                                                                                                                                                   \\ \cline{2-3} 
                               & SigmaSwiftStatistics & It is a collection of functions that perform statistical calculations in Swift. It can be used in Swift apps for Apple devices and in open source Swift programs on other platforms.                        \\ \cline{2-3} 
                               & SwiftyXMLParser      & Simple XML Parser implemented in Swift                                                                                                                                                                      \\ \hline
\multirow{7}{*}{Front-end}     & PinLayout            & Extremely Fast views layouting without auto layout. No magic, pure code, full control and blazing fast. Concise syntax, intuitive, readable \& chainable. PinLayout can layouts UIView, NSView and CALayer. \\ \cline{2-3} 
                               & FlexLayout           & Angular Flex Layout provides a sophisticated layout API using Flexbox CSS + mediaQuery.                                                                                                                     \\ \cline{2-3} 
                               & Cosmos               & This is a UI control for iOS and tvOS written in Swift.                                                                                                                                                     \\ \cline{2-3} 
                               & ImageSlideshow       & Customizable Swift image slideshow with circular scrolling, timer and full screen viewer.                                                                                                                   \\ \cline{2-3} 
                               & MarqueeLabel         & MarqueeLabel is a UILabel subclass adds a scrolling marquee effect when the text of the label outgrows the available width                                                                                  \\ \cline{2-3} 
                               & FontAwesome          & Use Font Awesome in your Swift projects                                                                                                                                                                     \\ \cline{2-3} 
                               & FlagKit              & Beautiful flag icons for usage in apps and on the web.                                                                                                                                                      \\ \hline
\end{tabular}




\end{center}


\chapter{Testing}

\chapter{Effort spent}


\end{document}
